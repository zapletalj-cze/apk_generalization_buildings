\documentclass[15pt]{article}
\usepackage[utf8]{inputenc}
%\renewcommand{\baselinestretch}{1.3}
\usepackage[czech]{babel}
\usepackage{xcolor}
\usepackage{comment}
\usepackage{amsmath}
\usepackage{amssymb}
\usepackage{algorithm}
\usepackage{algorithmicx}
\usepackage{gensymb}
\usepackage[margin=1in]{geometry}
\usepackage{setspace}
\usepackage{parskip}
\setlength{\parindent}{20pt}
\usepackage{fancyhdr}
\pagestyle{fancy}
\rhead{}
\lhead{Algoritmy počítačové kartografie}
\usepackage{graphicx}
\graphicspath{ {./images/} } 

\begin{document} 

% TITULNÍ STRÁNKA
\begin{titlepage}
        \centering
        {\large Přírodovědecká fakulta\par}
        {\large Univerzita Karlova\par}
        \vspace{1.5cm}
        \includegraphics[scale=0.4]{logo.png}\\
        \vspace{1.5cm}
        {\large Algoritmy počítačové kartografie}\\
        \vspace{0.2cm}
        {\large\textbf{Úkol č. 2: Generalizace budov}\par}
        \vspace{1.5cm}
        {\large Anna Brázdová a Petra Pajmová \par}
        \vspace{1.5cm}
        {\large 2.N-GKDPZ \par}
        {\large Praha 2024 \par}
        \vspace{1cm}
\end{titlepage}

\begin{spacing}{1.5}

% ZADÁNÍ 
\noindent\emph{Vstup: množina budov B =} \{$B_i$\}$_{i=0}^n$,\emph{budova} $B_i = $  \{$P_{i,j}$\}$_{j=1}^m$.

\noindent\emph{Výstup: $G(B_i)$}.

\noindent Ze souboru nečtěte vstupní data představovaná lomovými body budov a proveďte generalizaci budov do úrovně detailu LOD0. Pro tyto účely použijte vhodnou datovou sadu, např. ZABAGED, testování proveďte nad třemi datovými sadami (historické centrum měst, intravilán - sídliště, intravilán - izolovaná zástavba). 

\noindent Pro každou budovu určtete její hlavní směry metodami:

\begin{itemize}
  \item Minimum Area Enclosing Rectangle,
  \item PCA.
\end{itemize}

\noindent U první metody použijte některý z algoritmů pro konstrukci konvexní obálky. Budovu při generalizaci do úrovně detailu LOD0 nahraďte obdélníkem orientovaným v obou hlavních směrech, se středem v těžišti budovy, jeho plocha bude stejná jako plocha budovy. Výsledky generalizace vhodně vizualizujte.

\noindent Otestujte a porovnejte efektivitu obou metod s využitím hodnotících ktitérií. Pokuste se rozhodnout, pro které tvary budov dávají metody nevhodné vásledky, a pro které naopak poskytují vhodnou aproximaci.

\bigbreak

% insert table
\noindent\textbf{Hodnocení}
\bigbreak
\begin{center}
\begin{tabular}{|l|l|}

\hline
\textbf{Krok} & \textbf{hodnocení} \\ [0.5ex]  \hline\hline
Generalizace budov metodami Minimum Area Enclosing Rectangle a PCA. & 15 b. \\ \hline
\textit{Generalizace budov metodou Longest Edge.} & \textit{+ 5 b.} \\ \hline
\textit{Generalizace budov metodou Wall Average.} & \textit{+ 8 b.} \\ \hline
\textit{Generalizace budov metodou Weighted Bisector.} & \textit{+ 10 b.} \\ \hline
\textit{Implementace další metody konstrukce konvexní obálky.} & \textit{+ 5 b.} \\ \hline
\textit{Ošetření singulárních případů při generování konvexní obálky.} & \textit{+ 2 b.} \\ \hline
\textbf{Celkem} & \textbf{45 b.} \\ [0.5ex]  \hline

\end{tabular}
\end{center}



% POPIS A ROZBOR PROBLÉMU
\newpage
\section*{Popis a rozbor problému}































\newpage
\section*{Problematické situace a jejich rozbor}

\newpage
\section*{Popisy algoritmů formálním jazykem}

% pseudokód jarvis scan
\subsubsection*{Pseudokód metody Jarvis Scan}
    \begin{algorithm}
        \caption {\textit{Jarvis Scan}}
        \begin{algorithmic}[1]
            \State inicializuj konvexní obálku
            \State najdi počáteční bod $q$ s minimální hodnotou souřadnice $y$ 
            \State najdi počáteční bod $s$ s minimální hodnotou souřadnice $x$
            \State inicializuj poslední dva body
            \State přidej počáteční bod do konvexní obálky
            \State \textbf{dokud} $p_{j+1} \neq p_{min}$:
            \State \indent \textbf{pro} každý bod z množiny:
            \State \indent \indent vypočti úhel $\omega$
            \State \indent \indent $p_{j+1} = max \omega$
            \State \indent \indent přidej $p_{j+1}$ do konvexní obálky
            \State \indent \indent aktualizuj poslední dva body konvexní obálky
        \end{algorithmic}
    \end{algorithm}

% pseudokód winding number
\subsubsection*{Pseudokód metody Minimum Area Enclosing Rectangle}
    \begin{algorithm}
        \caption {\textit{Minimum Area Enclosing Rectangle}}
        \begin{algorithmic}[1]
            \State vytvoř konvexní obálku
            \State inicializuj $MMB$ a vypočítej jeho plochu $A$
            \State \textbf{pro} každou hranu konvexní obálky
            \State \indent vypočti úhel $\sigma$
            \State \indent otoč konvexní obálku o úhel $-\sigma$
            \State \indent spočti $MMB_{i}$ a vypočítej jeho plochu $A_{i}$
            \State \indent \textbf{pokud} $A_{i} >  A$:
            \State \indent \indent $A = A_{i}$
            \State \indent \indent $MMB$ = $MMB_{i}$
            \State \indent \indent $\sigma = \sigma_{i}$
            \State otoč $MMB_{i}$ o $\sigma_{i}$ a přeškáluj jeho plochu, tak, že $A_{i} =  A$
        \end{algorithmic}
    \end{algorithm}
    
% pseudokód longest edge
\subsubsection*{Pseudokód metody Longest Edge}
    \begin{algorithm}
        \caption {\textit{Longest Edge}}
        \begin{algorithmic}[1]
            \State inicializuj počet vrcholů $n$ a nejdelší hranu $l_{max} = 0$
            \State \textbf{pro} každou hranu:
            \State \indent vypočti délku hrany $l$
            \State \indent \textbf{pokud} $l > l_{max}$:
            \State \indent \indent $l_{max} = l$
            \State \indent \indent vypočti směrnici $l_{max}$ $\sigma$
            \State \indent otoč budovu o $ - \sigma$
            \State \indent zkonstruuj $MMB$ a vypočti jeho plochu $A$
            \State \indent otoč $MMB_{i}$ o $\sigma_{i}$ 
            \State \indent vypočti plochu budovy $A_{b}$ a přeškáluj plochu $MMB$ tak, že $A = A_{b}$
        \end{algorithmic}
    \end{algorithm}

% pseudokód longest edge
\subsubsection*{Pseudokód metody Wall Average}
    \begin{algorithm}
        \caption {\textit{Wall Average}}
        \begin{algorithmic}[1]
            \State vypočítej směrnici $\sigma$ pro libovolnou hranu
            \State inicializuj zbytek $r = 0$
            \State \textbf{pro} každou hranu:
            \State \indent vypočti směrnici $\sigma_{i}$
            \State \indent vypočti  $\Delta\sigma$
            \State \indent vypočti $k{i}$ a $r{i}$
            \State vypočti průměrný úhel $\sigma$:
            \State otoč budovu o $ - \sigma$
            \State zkonstruuj $MMB$ a vypočti jeho plochu $A$
            \State otoč $MMB$ o $\sigma$ 
            \State vypočti plochu budovy $A_{b}$ a přeškáluj plochu $MMB$ tak, že $A = A_{b}$
        \end{algorithmic}
    \end{algorithm}


\newpage
\section*{Vstupní data a výstupní data}

\newpage
\section*{Dokumentace}

\newpage
\section*{Závěr}

\newpage
\section*{Seznam literatury}


\end{spacing}
\end{document}